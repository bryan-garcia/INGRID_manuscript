\subsection{\label{sec:level2}Summary}
INGRID is introduced as a grid generator capable of producing grids for single-null (SNL), unbalanced double-null (UDN), and snowflake (SF) configurations. Current exported grids are in the UEDGE standard gridue file format, and future development anticipates the addition of additional formats as adoption of INGRID increases in the broader edge-plasma community. INGRID can be utilized via the INGRID Python package, or through a parameter file driven GUI mode. The internal equilibrium file analysis algorithm provides the ability to automatically identify the divertor configuration embedded within experimental data with minimal user interaction. A divide and conquer, geometry class hierarchy approach to grid generation is at the heart of INGRID and leads to the Patch map abstraction: a partition of the modeling domain that allows for localized grid generation. These localized grids are combined into a global grid that are then ready for export. Current computational scaling of grid generation algorithm follows a sublinear trend independent of magnetic-topology modeled. Benchmarking of INGRID against the internal grid generator in UEDGE was conducted on an SF75 configuration. These tests illustrated Ingrid's ability to both reproduce a grid obtained with UEDGE's internal grid generator and produce consistent UEDGE simulation results, but at a fraction of the effort with regards to obtaining a gridue file. 

\section{Acknowledgments}
This work was performed for U.S. Department of Energy by Lawrence Livermore National Laboratory under Contract DE-AC52-07NA27344.