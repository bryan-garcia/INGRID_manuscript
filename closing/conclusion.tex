\subsection{\label{sec:level2}Summary}
INGRID is a new grid generator for tokamak boundary region, it is
capable of producing grids for single-null (SNL), unbalanced
double-null (UDN), and snowflake-like (SF) configurations. Currently,
exported grids are in the format of the UEDGE code, as detailed in
Ref. (\cite{Rensink2017}); future development may include addition of
grid formats used by other codes if INGRID is adopted in the broader
edge-plasma community, beyond UEDGE. INGRID can be utilized via the
INGRID Python package, or through a parameter file driven GUI
mode. The internal equilibrium geometry analysis algorithm provides
the ability to automatically identify the divertor configuration
embedded within experimental data with minimal user interaction. A
divide-and-conquer, geometry class hierarchy approach to grid
generation is at the heart of INGRID and leads to the Patch map
abstraction: a partition of the modeling domain that allows for
localized grid generation. These localized grids are combined into a
global grid that are then ready for export. Current computational
scaling of grid generation algorithm follows a sublinear trend
independent of magnetic-topology modeled. Benchmarking of INGRID
against the internal grid generator in UEDGE is demonstrated for an
SF75 snowflake-like configuration. These tests illustrate INGRID's
ability to produce practical grids for tokamak edge modeling, for
complex magnetic flux function with one or two X-points in the domain,
and for nontrivial target plate geometry.


\section{Acknowledgments}
The authors would like to thank M.E.Rensink for his help with grid
generation in UEDGE. This work was performed for U.S. Department of
Energy by Lawrence Livermore National Laboratory under Contract
DE-AC52-07NA27344, and General Atomics under Contract DE-FG02-95ER54309.
