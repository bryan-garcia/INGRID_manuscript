\subsection{\label{sec:level2}The INGRID parameter file}
INGRID has been designed to be controlled from a single configuration/parameter file when operating in GUI mode. We have decided to use YAML formatted files for the parameter file\textbf{[CITE YAML]}. This YAML file is similar to the familiar Fortran namelist files due to the key-value structure it is based off of. YAML is in an easy to read format that has extensive support within Python. With the PyYAML library, Python reads a YAML formatted file and internally represents it as a Python dictionary. This allows users to model cases in the INGRID GUI and reuse a parameter file in scripting for later usage (e.g. batch grid generation). Some key controls within the parameter file include: EFIT file specification, specification of number of x-points, approximate coordinates of x-point(s) of interest, approximate magnetic-axis coordinates, psi level values, and target plate settings (files, transformations). Other controls in the parameter file include: path specification for data files, grid cell np/nr values, poloidal and radial grid transformation settings, limiter specific settings, saving/loading of Patch maps, gridue settings, and debug settings. This is not an exhaustive list. Further details can be found in INGRID's Read The Docs online documentation. Section A in the appendix shows a small snippet of an INGRID parameter file for an SNL case. 