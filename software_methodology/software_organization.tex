\subsection{\label{sec:label2}INGRID package organization}
The top-level INGRID package includes a collection of modules responsible for the essential grid-generation processes, as well as two subpackages dedicated to the GUI and supported magnetic-topology models. Modules \texttt{ingrid} and \texttt{utils} contain classes responsible for management and execution of INGRID functionality, whereas modules \texttt{geometry}, \texttt{interpol}, and \texttt{line\_tracing} contain classes responsible for computational tasks such as topology modeling, interpolation of data, and line tracing. \\ \indent
The \texttt{Ingrid} class is contained within the \texttt{ingrid} module and is designed to provide a high-level interface for users. This \texttt{Ingrid} class is used to activate INGRID's GUI mode and also contains high-level methods for importing data, visualizing data, analyzing data, grid-generation, and exporting of data; all of which can be utilized in Python scripts. Class \texttt{IngridUtils} is contained within the \texttt{utils} module and serves as the base class for \texttt{Ingrid}. \texttt{IngridUtils} class methods encapsulate much of the lower-level software details used to implement the methods in the \texttt{Ingrid} class. Because of this, \texttt{IngridUtils} is encouraged for use by advanced users and developers of INGRID. In addition to \texttt{IngridUtils}, class \texttt{TopologyUtils} can be found within the \texttt{utils} module. In a manner similar to \texttt{IngridUtils}, the \texttt{TopologyUtils} class serves as a base class for each magnetic-topology class within the \texttt{topologies} subpackage. \texttt{TopologyUtils} contains key methods for generating Patch-Maps, visualizing data, generating grids, and exporting grids in gridue format. Eight magnetic-topology classes are contained within their own modules within the \texttt{topologies} subpackage: \texttt{SNL}, \texttt{UDN}, \texttt{SF15}, \texttt{SF45}, \texttt{SF75}, \texttt{SF105}, \texttt{SF135}, and \texttt{SF165}. Each magnetic-topology class contains configuration specific line-tracing instructions for construction of Patch-Maps, Patch-Map layout information, and gridue formatting information. \texttt{Ingrid} and \texttt{IngridUtils} conduct analysis of MHD equilibrium data in order to decide which magnetic-topology class to instantiate from the \texttt{topologies} subpackage. The \texttt{IngridUtils} class always maintains a reference to the instantiated object in order to effectively manage grid-generation.\\ \indent
All GUI operation is managed by class \texttt{ingrid\_gui} within the \texttt{gui}s subpackage. INGRID's GUI front-end was developed with the tkinter package; a Python interface to the Tk GUI toolkit that is available within the Python Standard Library. Class \texttt{ingrid\_gui} is simply responsible for managing event handling, and managing an \texttt{Ingrid} object that is used to drive the GUI with direct calls to the available high-level methods.\\ \indent
Beyond modules \texttt{ingrid} and \texttt{utils}, modules \texttt{geometry}, \texttt{interpol}, and \texttt{line\_tracing} form the computation and modeling foundation of INGRID. Classes \texttt{Bicubic} and \texttt{EfitData} can be found within the \texttt{interpol} module. The \texttt{Bicubic} class handles biciubic interpolation of data and is composed with classes that directly utilize bicubic interpolation. Class \texttt{EfitData} is used to provide an interpolated representation of provided MHD equilibrium data. \texttt{EfitData} computes partial derivative information of MHD equilibrium data, provides interpolated $\psi$ function values by interfacing with class \texttt{Bicubic}, and contains methods for visualisation of interpolated MHD equilibrium data. Module \texttt{geometry} contains classes \texttt{Point}, \texttt{Line}, \texttt{Patch}, and \texttt{Cell}. These classes are the building-blocks for creation of Patch-Maps and generation of grids.