\subsection{\label{sec:level2}INGRID workflow}
INGRID can be operated via GUI or utilizing the INGRID library directly in Python scripts. The GUI workflow highlights the interactive nature of INGRID by allowing users to visually inspect MHD equilibrium data, configure geometry, and refine parameter file values on the fly. For both GUI operation and scripting with INGRID, the high-level INGRID workflow is: (i)  Parameter file visualization and editing, (ii) Analysis of MHD equilibrium data and creation of Patch map, (iii) Patch map refinement and gridue export.
\noindent
INGRID internally handles step (ii) and leaves the user to with steps (i) and (iii). These steps are where the user is able to customize the Patch map and grid to meet their modeling needs.\\ \indent
Step (i) in the INGRID workflow allows users to visually inspect MHD equilibrium data, target-plates \& limiter geometry, and psi-level contours that are specified within a loaded parameter file. Since creation of grids is tied directly to MHD equilibrium analysis and Patch map creation, step (i) is crucial for successful grid generation. To simplify this step, the INGRID GUI provides an easy to use workspace for preparation of a parameter file for the subsequent analysis of MHD equilibrium data and Patch map creation. Examples of common operations at this step include modifications to strike-point geometry and psi-level boundaries for subsequent Patch maps. Once a user is satisfied with parameter file settings, step (ii) can be immediately executed with no further user intervention. Should any errors in Patch map creation occur (e.g. misplaced target-plates, psi-boundaries that do not conform to configuration specific requirements), INGRID will prompt the user and allow for appropriate edits to be made. Upon completion of step (ii), the created Patch map will be provided to users as a new matplotlib figure. From here the user can decide to proceed with Patch map refinement or start over at step (i) to make edits to the Patch map.\\ \indent
In order to streamline grid generation and skip directly to step (iii), INGRID supports Patch map reconstruction. This feature allows users to bypass line-tracing routines by reloading a saved Patch map from a previous INGRID session. To do so, INGRID encodes essential geometry and topology analysis data in a specially formatted dictionary that is then saved as a NumPy binary file. Class \texttt{IngridUtils} handles the encoding and reconstruction of Patch maps. These reconstruction features can be configured by the user within the INGRID parameter file.\\ \indent
After a Patch map has been generated or reconstructed, users can configure grid generation specific settings that will be utilized during Patch map refinement. Similar to Patch map generation, once all local subgrids have been created within Patch objects, a new matplotlib figure is presented with the generated grid. From here, users can make grid generation setting edits in the INGRID parameter file or proceed to exporting a gridue file.
% \subsection{\label{sec:level2}Obtaining and running INGRID}
% INGRID is currently available for download on Github and extensive documentation has been prepared for both installation and running of the code. We refer the interested reader to our online INGRID Read The Docs.