\subsection{\label{sec:level2}Choice of development in Python}

INGRID has been exclusively developed in the Python programming
language. Python has an advantage as a free, community supported
language with extensive graphical and numerical libraries. In addition
to the developer tools readily available, the large scientific
computing user base puts INGRID in a good position for extensive use
within the edge-plasma modeling community and continued
development. Python is a choice of object-oriented programming
language that is being increasingly utilized in major tokamak plasma
modeling projects such as OMFIT \cite{Meneghini_2015,
  Orso_MENEGHINI2013} and PyUEDGE \cite{PyUEDGE}. This makes Python a
natural and consistent choice to carry out software development
needs. INGRID makes heavy use of Python's standard computational and graphical libraries such as numpy, SciPy, matplotlib, SymPy, PyYAML.
and tkinter \cite{numpy_5725236, virtanen2019scipy,
  matplotlib_4160265, PyYAML}. Software project setup, maintenance, and
installation is easily achieved in Python through native package
structuring conventions and external package management tools such as
setuptools.  Within the INGRID package are a variety of modules
containing classes related to geometric representations, line tracing,
interpolation, and other INGRID management tasks. Two INGRID
subpackages are dedicated to GUI development and magnetic topology
modules.
