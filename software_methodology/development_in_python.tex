\subsection{\label{sec:level2}Choice of development in Python}

INGRID has been exclusively developed in the Python programming
language. Python has an advantage as a free, community supported
language with extensive graphical and numerical libraries. In addition
to the developer tools readily available, the large user base puts
INGRID in a good position for heavy use within the community. Since
Python is an object-oriented programming language that is being
increasingly utilized in major tokamak plasma modeling projects such
as OMFIT \cite{Meneghini_2015, Orso_MENEGHINI2013}) and PyUEDGE
\textbf{[CITE PyUEDGE?]}, Python was the natural choice to carry out
our development needs. We made heavy use of Python's free and robust
libraries in our computational tasks, data visualization, and GUI
development. Packages such as numpy, SciPy, matplotlib, SymPy, PyYAML,
and tkinter are indispensible developer tools utilized within the
INGRID project\cite{numpy_5725236, virtanen2019scipy,
  matplotlib_4160265}. Python being a object-oriented programming
language was also a key factor in the decision since this allowed us
to easily translate the abstractions discussed earlier into a Python
INGRID package. Within the INGRID package are a variety of modules
containing classes related to geometric representations, line tracing,
interpolation, and other INGRID management tasks. We have two
subpackages dedicated to GUI development and magnetic topology
modules.
