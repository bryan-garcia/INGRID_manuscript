\subsection{\label{sec:level2} Overview}
We now discuss user operation of the INGRID code. INGRID can be operated via it's GUI or through incorporation in Python scripts by utilizing the INGRID package directly. We will provide an overview of the GUI workflow since it highlights the interactive nature of INGRID. Writing scripts with the INGRID package is beyond the scope of this report. For both GUI operation and scripting with INGRID, the high-level INGRID workflow is as follows:
\begin{enumerate}
    \itemsep-4pt
    \item Prepare configuration file for INGRID
    \item Identify divertor configuration and generate the appropriate Patch map
    \item Create and export a grid from the obtained Patch map
\end{enumerate}
\noindent
INGRID internally handles step 2 (as noted earlier in the report) and leaves the user to with steps 1 and 3. These steps are where the user is able to customize the Patch map and grid to meet their modeling needs.

\subsection{\label{sec:level2}The INGRID parameter file}
INGRID has been designed to be controlled from a single configuration/parameter file when operating in GUI mode. We have decided to use YAML formatted files for the parameter file\textbf{[CITE YAML]}. This YAML file is similar to the familiar Fortran namelist files due to the key-value structure it is based off of. YAML is in an easy to read format that has extensive support within Python. With the PyYAML library, Python reads a YAML formatted file and internally represents it as a Python dictionary. This allows users to model cases in the INGRID GUI and reuse a parameter file in scripting for later usage (e.g. batch grid generation). Some key controls within the parameter file include: EFIT file specification, specification of number of x-points, approximate coordinates of x-point(s) of interest, approximate magnetic-axis coordinates, psi level values, and target plate settings (files, transformations). Other controls in the parameter file include: path specification for data files, grid cell np/nr values, poloidal and radial grid transformation settings, limiter specific settings, saving/loading of Patch maps, gridue settings, and debug settings. This is not an exhaustive list. Further details can be found in INGRID's Read The Docs online documentation. Section A in the appendix shows a small snippet of an INGRID parameter file for an SNL case. 

\subsection{\label{sec:level2}Obtaining and running INGRID}
INGRID is currently available for download on Github and extensive documentation has been prepared for both installation and running of the code. We refer the interested reader to our online INGRID Read The Docs.