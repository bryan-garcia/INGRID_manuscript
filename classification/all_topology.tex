To understand the range of geometric possibilities in presence of one
and two X-points, consider the diagrams in Fig.(\ref{fig:all_conf}).

If there is a single X-point in the region it defines a separatrix,
Fig. (\ref{fig:all_conf} (a)). The X-point is a self-intersection
point of the separatrix that divides the plane into three
topologically distinct regions. One is the ``core plasma region'',
containing the magnetic axis, or the O-point. Next, there is a region
lying opposite to the core plasma region, across the X-point, and it
is called the ``private flux region'', or PFR. And the remaining part
is the ``common flux region'', or scrape-off-layer (SOL) region.

A second X-point can be added outside of the core plasma region. Ignoring the degenerate case when the secondary X-point is on the primary-separatrix, there are two topologically
distinct possibilities: with respect to the primary separatrix the
second X-point can be either in the private flux region,
Fig. (\ref{fig:all_conf} (c)); or in the common flux region
Fig. (\ref{fig:all_conf} (b)). The former case is called
``snowflake-plus'' and the latter case is called ``snowflake-minus''
\cite{Ryutov2007}.

However, from the grid generation perspective, we consider further
variations of ``snowflake-like'' configurations, as shown in
Fig. (\ref{fig:all_conf}). Consider a line orthogonal to flux surfaces
and passing through the secondary X-point $X_2$, and consider the
intersection of this line with the primary separatrix,
$X^{\prime}_2$. For ``snowflake-minus'' configuration there five
possibilities for the projection point $X^{\prime}_2$. It can belong
to: (i) the arc $[M_W,M_E]$ connecting the two midplane-level points,
or (ii) the arc $[M_W,X_1]$, or (iii) the arc $[M_E,X_1]$, or the arc
(iv) $[X_1,S_W]$ connecting the primary X-point and one of the
``strike-points'', or (v) the arc $[X_1,S_E]$. For ``snowflake-plus''
configuration, there are two possibilities for the projection point
$X^{\prime}_2$. It can belong to: (i) $[X_1,S_E]$, or to (ii)
$[X_1,S_W]$.

Based on the location of the secondary X-point $X_2$ and its
orthogonal projection on the primary separatrix $X_2^{\prime}$, we use
the following notation for the configurations with two X-points:

\begin{itemize}
	\item UDN: $SF-$, $X^{\prime}_2 \in [M_E,M_W]$
	\item SF15: $SF-$, $X^{\prime}_2 \in [M_E,X_1]$
	\item SF165: $SF-$, $X^{\prime}_2 \in [M_W,X_1]$
        \item SF45: $SF-$, $X^{\prime}_2 \in [S_E,X_1]$
        \item SF135: $SF-$, $X^{\prime}_2 \in [S_W,X_1]$
        \item SF75: $SF+$, $X^{\prime}_2 \in [S_E,X_1]$
        \item SF105: $SF+$, $X^{\prime}_2 \in [S_W,X_1]$
\end{itemize}

The notation for snowflake-like configurations is inspired by local
geometric analysis of near-snowflake configurations and the numbers
correspond to the geometric angle defining the position of the
secondary X-point \cite{Ryutov2010}. Our geometric classification here
is topology-based, it applies to more general situations when the two
X-points can be far apart; but the notation introduced in literature\cite{Ryutov2010} is still useful. All in all, with the
single-null (SNL) configuration included, for either one and two
X-points in the domain, there are eight possible configurations. We
don't consider degenerate cases where the secondary X-point is exactly
on the primary separatrix, or the projected point $X^{\prime}_2$ is
exactly on the primary X-point $X_1$, or it is exactly on a midplane
point $M_E$ or $M_W$; the assumption is that a practical, experiment-relevant configuration
would always have some finite degree of asymmetry to fall into one of
the eight considered categories.
