\subsection{\label{sec:level2}Tokamak edge plasma modeling}
%
Research in tokamak edge plasma physics is critical for realizing
practical fusion energy and designing future fusion reactors. One of
the greatest challenges that tokamak edge plasma researchers face
today is determining effective methods for controlling particle and
heat fluxes on tokamak plasma-facing components (PFC). A possible
solution is the use of advanced divertor configurations. \\ \indent
%
The traditional X-point divertor configuration uses a first-order null
point for the poloidal field, which is usually placed at the bottom
(``Lower Single Null'', or LSN) or at the top (``Upper Single Null'',
or USN).  The traditional double-null configuration adds another
first-order X-point, on the opposite end; if the two X-points are on
the same flux surface this configuration is often called
``Double-Null'' or DN, otherwise it is an ``Unbalanced Double-Null''
or UDN. \\ \indent
%
In contrast, several advanced divertor configurations have been
proposed where a a secondary x-point is included in the divertor
region. These include snowflake-like configurations with a secondary
X-point close to a primary one, and the X-point Target configuration
where a secondary X-point is placed in the divertor leg near the
target plate \cite{}. \\ \indent
%
Snowflake-like configurations approximate a configuration with an
exact second-order null of the poloidal field dubbed ``snowflake''
\cite{Ryutov2007}. In practice, instead of an exact second-order null,
a configuration is used where two regular X-points are brought close
together \cite{Ryutov2008}. On the other hand, for an X-Point Target
configuration \cite{LaBombard2015}, a secondary X-point is introduced
far away from the primary X-point, in the divertor leg near the target
plate. \\ \indent
%
Each of these divertor configuration is characterized by locations of
the primary and secondary x-points in the domain. The primary X-point
is the most significant as it separates the plasma into the hot core
region and colder scrape-off layer (SOL) region. However, a secondary
X-point helps redirect and distribute the flux of plasma particles and
energy over multiple locations (strike points) on the material
surface.  Furthermore, a secondary X-point helps increase plasma
radiation in the divertor, and potentially it may cause other
interesting and important effects on divertor plasma physics.
