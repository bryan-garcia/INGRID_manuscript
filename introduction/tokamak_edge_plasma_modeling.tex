\subsection{\label{sec:level2}Tokamak edge plasma modeling}
Research in tokamak edge plasma physics is essential for realizing practical fusion energy and designing future fusion reactors. One of the greatest challenges that tokamak edge plasma researchers face today is determining effective methods for controlling particle and heat fluxes on tokamak plasma-facing components (PFC). A possible solution is the use of advanced divertor configurations beyond that of the standard configurations. These standard divertor configurations are characterized by a single magnetic null-point (the primary x-point) in the tokamak domain \cite{krash1}, whereas advanced divertor configurations include a secondary x-point in the tokamak domain of interest \cite{GINGRID_nontrivial_1}. In the case of two x-points in the domain, one obtains configurations such as the ``Unbalanced Double-Null" (UDN), ``Snowflake-Minus" (SF-), and ``Snowflake-Plus" (SF+) \cite{GINGRID_nontrivial_2, Ryutov_2010}; and also the X-Divertor and X-Point Target Divertor configurations. Each divertor configuration depends on the locations of the primary and secondary x-points in the domain. The simplest divertor configuration consists of a single x-point (primary x-point) in the $(r,z)$ plane of our toroidally symmetric tokamak. Depending on the case of interest, the primary x-point can be located anywhere in the $(r,z)$ plane. We refer to these cases as with a single x-point as ``single-null" (SNL) configurations, and further classify them as ``lower single-null" (LSN) or ``upper single-null" (USN) configurations depending on which side of the horizontal-midplane intersecting the magnetic axis we find the x-point resides in. The primary x-point is significant in SNL cases as it corresponds to the magnetic separatrix which divides the domain into regions \cite{LLNL-TR-731515} commonly referred to in practice as the ``scrape-off layer" (SOL), ``private-flux region" (PF), and ``core-plasma region" (CORE).