\subsection{\label{sec:level2}Tokamak edge plasma modeling}
%
Research in tokamak edge plasma physics is critical for realizing
practical fusion energy and designing future fusion reactors. One of
the greatest challenges that tokamak edge plasma researchers face
today is determining effective methods for controlling particle and
heat fluxes on tokamak plasma-facing components (PFC). A possible
solution is the use of advanced divertor configurations
\cite{Kotschenreuther2007}.

The traditional X-point divertor configuration uses a first-order null
point for the poloidal field, which is usually placed at the bottom of
the core plasma, or at the top. The traditional double-null
configuration uses one first-order X-point at the bottom and one at
the top.

In contrast, several advanced divertor configurations have been
proposed where a a secondary x-point is included in the divertor
region. These include snowflake-like configurations and the X-point
Target configuration. Snowflake-like configurations approximate a
configuration with an exact second-order null of the poloidal field
dubbed ``snowflake'' \cite{Ryutov2007}. In practice, instead of an
exact second-order null, a configuration is used where two regular
X-points are brought close together, which leads to snowflake-plus and
snowflake-minus configurations \cite{Ryutov2008}. On the other hand,
for an X-Point Target configuration \cite{LaBombard2015}, a secondary
X-point is introduced in the divertor far away from the primary
X-point, in the divertor leg near the target plate.

Each of these divertor configuration is characterized by locations of
the primary and secondary X-points in the domain. The primary X-point
is the most significant as it separates the plasma into the hot core
region and colder scrape-off layer (SOL) region. However, a secondary
X-point helps redirect and distribute the flux of plasma particles and
energy over multiple locations (strike points) on the material
surface.  Furthermore, a secondary X-point may help increase plasma
radiation in the divertor, and potentially it may cause other
interesting and important effects in divertor plasma.

%%%%%%%%%%%%%%%%%%%%%%%%%%%%%%%%%%%%%%%%%%%%%%%%%%%%%%%%%%%%%%%%%%%%%%%%%%%%%%%%

