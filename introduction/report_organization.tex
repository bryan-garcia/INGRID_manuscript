\subsection{\label{sec:level2}Report organization}
This report is organized into three sections. Section 2 contains a discussion on the design choices made for the INGRID project. We elaborate on the object-oriented ``geometry hierarchy" that allows for robust grid generation within INGRID. We detail INGRID specific nomenclature and it's motivation is. From here, the central divertor configuration identification algorithm is able to be explained. We then introduce the reader to the Patch map, and it's key role in obtaining a final grid.
In section 3, an overview of the INGRID parameter file and GUI workflow is provided. This section consists of brief glimpses at the example cases provided with the public release of INGRID. These examples demonstrate the simplicity of generating of a grid in INGRID for both an SNL and SF case. A detailed explanation can be found in INGRID's Read The Docs. We close section 3 with a comparison of two UEDGE simulations that are from an INGRID generated grid and a grid generated from UEDGE's internal grid generator.