\subsection{Grid generation for tokamak edge plasma simulations}

Tokamak boundary and divertor plasma modeling relies heavily on edge
transport modeling codes such as UEDGE \cite{Rognlien1999}, SOLPS
\cite{Wiesen2015}, EDGE2D \cite{Simonini1994}, just to mention a few
major ones. These codes are similar in many ways; they all solve the
time-evolution fluid equations for toroidally-symmetric, collisional
plasma based on the Braginskii equations, using ad-hoc radial
transport coefficients.The simulations are usually carried out in the
actual geometry of a modeled tokamak, to account for details of
magnetic field geometry and plasma-facing components.

Computational grids for tokamak boundary plasma modeling are usually
chosen to follow the magnetic flux surfaces, to avoid numerical
pollution caused by the extreme anisotropy of plasma transport along
and across the magnetic field \cite{Umansky2005}. Thus the
computational grid has to follow the underlying magnetic field, and
with one or several X-points present in the simulation domain the grid
topology can become highly nontrivial.

There are several grid generators for tokamak edge plasma region
currently in use. The UEDGE code uses a grid generator that is a part
of the UEDGE package, SOLPS normally uses grid generator CARRE
\cite{Marchand1996}, and EDGE2D usually relies on grid generator GRID2D
\cite{Taroni1992}. These are sufficient in most cases for modeling
single-null and double-null configurations. However modeling of
advanced divertors may require incorporating secondary X-points in the
divertor region, and grid generators currently in use for the major
edge transport codes are not inherently designed to produce
computational grids for general configurations containing more than
one X-point at arbitrary locations in the domain.

To increment the capabilites for grid generation for tokamak boundary
plasma modeling, in particular for advanced divertors, a new grid
generator INGRID has been developed, as described in the present
report.  INGRID (Interactive Grid Generator) is a Python based,
interactive, grid generator for edge plasma modeling that is capable
of handling configurations with one or two X-points anywhere in the
computational domain. INGRID provides a robust set of tools such as an
easy to use GUI intended for users of all levels. By internally
handling the challenges that typically arise with generating grids for
tokamak edge plasma region, INGRID can indeed improve efficiency in a
user's workflow for edge-plasma modeling. The INGRID algorithm's
inspiration was drawn from an older IDL-based project GINGRED
\cite{Izacard2017} where the ``divide and conquer'' strategy for
grid generation was first tried. An important motivating factor for
implementing INGRID in Python was using an open-source language with
highly advanced numerical and graphical libraries.

%%%%%%%%%%%%%%%%%%%%%%%%%%%%%%%%%%%%%%%%%%%%%%%%%%%%%%%%%%%%%%%%%%%%%%%%%%%%%%%%

