\subsection{Grid customization}

INGRID provides users a number of tools for customization of generated
grids, which can be controlled via the parameter file. Users can
modify both the Patch-Map and the subgrids in order to optimize the
global grid.

Default settings for constructing a Patch-Map use the horizontal plane
through the magnetic axis, commonly referred to as the midplane.
However, INGRID allows shifting the magnetic axis vertically and
horizontally. Moreover, instead of using the horizontal direction for
the midplane, the user can set two angles defining a ``generalized
midplane''.

Also, for those patches that include an X-point as one of their
vertices, the default poloidal boundaries use in the East and West
directions from the X-point along the $\grad \Psi$ direction. However,
the user can redefine those curved, replacing them by straight lines
in a desired direction.

For fine-tuning the grid, subgrids can be adjusted as well. By
default, during Patch refinement, grid seed-points are distributed
uniformly in length along the radial boundaries and
distributed uniformly along the poloidal boundaries in
locally-normalized $\Psi$. This default behavior can be changed so
that grid seed-point placement obeys a user specified distribution
function.

Another important customization feature in INGRID is a
``\texttt{distortion\_correction}" tool for mitigating grid
shearing. This tool allows the user to set limits on deviation between
the poloidal direction and the $\grad \Psi$ direction. This
effectively allows enforcing nearly orthogonal subgrids in selected
patches. If the ``\texttt{distortion\_correction}'' constraint cannot
be satisfied (vertex leaves the Patch), INGRID will backtrack until
the vertex is within the Patch bounds. An example of the
\texttt{distortion\_correction} feature applied to a grid can be seen
in Fig.( \ref{fig:distortion_correction}).
