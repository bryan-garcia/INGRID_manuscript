\subsection{Construction of grids}
INGRID employs a ``divide and conquer" approach to grid generation by stitching together a collection of local-grids generated by each individual Patch within the Patch map. That is, each Patch is responsible for generating a ``subgrid" for the region of the computational domain it represents. The act of ``refining" a Patch into a subgrid is handled entirely by each Patch object itself.\\
Patch objects are quadrilaterals defined entirely by four Line objects. Each Line object used to define a Patch is saved as the class attribute ``N", ``S", ``E", or ``W" in order to resolve which face of the quadrilateral each Line object corresponds to. The INGRID convention is to define ``N" and ``S" Patch boundaries to be Line objects along constant $\hat{\Psi}$ surfaces, and define ``E" and ``W" Patch boundaries to be Line objects connecting ``N" to ``S" and ``S" to ``N" respectively. E/W Patch boundaries are not required to be orthogonal to $\hat{\Psi}$ surfaces. INGRID generates non-orthogonal grids as a result. However, INGRID has tools for making grids close to locally-orthogonal in desired Patches, as discussed further. A clock-wise orientation along the Patch boundary is chosen to be an INGRID convention. This is done by defining ``W" to be strictly increasing in the standard normalized $\hat{\Psi}$ and ``E" to be strictly decreasing in $\hat{\Psi}$. These conventions are essential for INGRID visualization methods (e.g. filling of Patches during plotting), and the refinement of a Patch into a subgrid.  \\
With a properly defined Patch object, INGRID is capable of generating an $\text{np} \times \text{nr}$ local subgrid with line tracing. Users can specify the number of grid cells to generate in the poloidal direction (np) and radial direction (nr) for each Patch individually in the INGRID parameter file so long as the choices are consistent in index-space. To create a subgrid, INGRID first generates ``seed-points" along the ``E" and ``W" Patch boundaries. This is done by creating splines along the ``E" and ``W" Patch boundaries that are then parameterized in $\hat{\Psi}$. Parameterization of E/W Patch boundaries is done by first computing locally-normalized $\hat{\Psi}$ data (zero on S boundary endpoints, one on N boundary endpoints) and utilizing the locally normalized $\hat{\Psi}$ data to create splines along the E/W Patch boundaries. This parameterization in locally-normalized $\hat{\Psi}$ allows INGRID to divide the E/W Patch boundary according to $\hat{\Psi}$ level. Once creation of the parameterized splines is complete, INGRID generates ``seed-points" along the E/W boundary according to the nr value provided. These seed-points are used as the initial point for line-tracing along it's corresponding $\hat{\Psi}$ surface. The LineTracing class generates a total of nr$-2$ Line objects between the N/S Patch boundaries along poloidal surfaces. The seed-points generated for the LineTracing class are saved as vertices to be used in the completed local subgrid.\\
In addition to vertices generated along E/W Patch boundaries, INGRID generates vertices along N/S Patch boundaries and the Line objects generated from the E/W Patch boundary vertices/seed-points. Splines are utilized to model the N/S Patch boundaries and the generated intermediate Line objects, but INGRID this time parameterizes the curves in normalized length. This parameterization in length allows INGRID to divide the N/S Patch boundaries and intermediate Line objects in length according to the np value provided by the user. At each subdivision along the splines representing poloidal surfaces, a vertex is placed. A total of $(nr+1)(np+1)$ many vertices are generated during Patch refinement to create the final subgrid.\\
INGRID finalizes Patch refinement within a single Patch by utilizing the generated vertices to define local subgrid mesh. The subgrid contained within a Patch is simply a collection of Cell objects instantiated from the generated vertices. Each Cell object is defined by four Point objects corresponding to the vertices. The Cell center is computed upon instantiation. 
The Patch refinement process repeats for each Patch in a Patch map. Before export of a gridue file, INGRID performs a sweep over the collection of subgrids to refine alignment of grid cells among adjacent Patch objects and x-points. After internal grid adjustment is finalized, the gridue file data is ready for export.