\subsection{Grid construction}

For a given magnetic configuration, a Patch-Map represents a very
crude grid where each Patch is a ``quadrilateral'' with four vertices.
The radial sides of this quadrilateral are defined by two flux
surfaces, $\Psi(R,Z)=\Psi_1$ and $\Psi(R,Z)=\Psi_2$. The poloidal
sides of a Patch are usually constructed to be aligned with $\nabla
\Psi$, which makes the skeleton grid locally orthogonal. 

However, more generally the poloidal sides of a Patch can deviate from
the $\nabla \Psi$ direction. For example, for those patches that
contain the poloidal boundaries of the domain, given by the target
plates, one of the sides is defined by the target plate shape. The
curve describing the target plate can be arbitrary, as long as it does
not form ``shadow regions'', i.e., $\Psi$ is a continuous function of
the length along the plate.

Going beyond the skeleton grid, a Patch can be divided in a number of
radial and poloidal zones, forming a subgrid local to this patch. The
radial zones are constructed to be aligned with flux surfaces, so the
global grid remains aligned with the poloidal magnetic field. For the
poloidal zones, the main algorithm is based on dividing the patch
poloidally into equal size segments. However, as described further,
there are options in the code for controlling the radial and poloidal
distribution of subgrid.

The radial and poloidal dimensions and distribution of subgrid on a
given Patch are not entirely independent of subgrids on other patches
as the global grid still has to be Cartesian in the index space. Thus
the poloidal grid has to be consistent for those patches that are
stacked on top of each other radially, and the radial grids have to be
consistent for those patches stacked on top of each other poloidally.
