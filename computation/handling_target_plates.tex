\subsection{Handling of target plates}
Users can specify target-plates and/or limiter geometry files for their edge-plasma modeling needs. INGRID utilizes specially formatted data files (text and numpy binary) in order to define strike-point geometry for modeling. All strike-point geometry files are to be provided via the INGRID parameter file. Each individual strike-point geometry item of interest is defined by it's own data file.
For use of text formatted geometry files, users must specify $(r,z)$ coordinates on each line within the geometry file. Python formatted single-line comments are also supported by INGRID. Examples of text formatted geometry files can be seen in the appendix. 
For use of numpy-binary files, users must also adhere to a particular internal file structure. Given two numpy arrays of shape $(n, )$ representing $r$ and $z$ coordinate values respectively, one can define a numpy array of shape $(2,n)$ representing the $n$-many points required to model a strike-point geometry of interest. This numpy array of shape $(2, n)$ can be saved into a numpy binary file in order to be loaded into INGRID.
In addition to the requirements above, INGRID asserts that strike-point geometry files used for Patch maps are monotonic in psi (i.e. no shadow regions). This assertion allows for INGRID generated grids to conform exactly to a user's specified strike-point geometry. While operating INGRID in GUI mode, users will be warned if the loaded geometry file is not monotonic in psi. 