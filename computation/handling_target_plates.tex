\subsection{Handling of target plates}
INGRID users must specify the geometry of the limiter and/or target plates, to represent the shape of material walls in a modeled device. The limiter and target plates are represented in INGRID by a piecewise-linear model defined by a set of nodes; the (r,z) coordinates of those nodes are expected to be provided in separate data files. There is one data file for the limiter and one for each target plate, either in the text format or as a NumPy binary. The names of those data files are set in the INGRID parameter file. In the case that the limiter and target plate data are provided in text format, the user must specify (r,z) coordinates for each point defining the surface sequentially on a separate line in the corresponding data file; and Python-formatted single- line comments can be included, as shown in the Appendix.
For use of NumPy binary files, users must also adhere to a particular internal file structure. Given two NumPy arrays of shape $(n, )$ that represent $r$ and $z$ coordinate values respectively, one can define a NumPy array of shape $(2,n)$ representing the $n$-many points required to model the piecewise-linear model of interest. This NumPy array of shape $(2, n)$ can be saved into a NumPy binary file in order to be loaded into INGRID.
In addition to the requirements above, INGRID asserts that strike-point geometry files used for Patch maps are monotonic in psi (i.e. no shadow regions). This assertion allows for INGRID generated Patch maps to conform entirely to a user's specified strike-point geometry. While operating INGRID in GUI mode, users will be warned if the loaded geometry file is not monotonic in psi. 