\subsection{Patch map construction}

For a given magnetic field, after the analysis of magnetic geometry
and establishing what configuration corresponds to it, INGRID creates
a corresponding Patch map which is an ordered collection of
quadrilateral patches defining the skeleton grid. A patch map is
constructed using numerical line tracing of poloidal and radial
surfaces through the X-points, and radial and poloidal surfaces
defining the domain boundaries. Radial domain boundaries are flux
surfaces correspondging to maximum and minimum values of the poloidal
flux function $\Psi$, these values are defined by the user in the
input file. The poloidal domain boundaries are target plates surfaces,
also defined in the input file.

For each of the divertor configurations that INGRID can use - which
includes a single-null, unbalanced double-null, and six snowflake-like
configurations - there is a specific type of patch map that defines
the topology of this configuration. For example, for a SNL
configuration shown in Fig. (), a patch map includes two radial zones,
two poloidal zones defining divertor legs, and four poloidal zones
defining the edge plasma domain around the last closed flux
surface. Such a patch map that contains twelve patches is necessary
and sufficient to represent a general single-null geometry, albeit in
a most basic and crude way. Furtermore, a finer grid representing a
single-null geoemtry can be always represented as this patch map with
local refinement applied to one or several of these twelve patches.
For a more complicated unbalanced double null (UDN) geometry shown in
Fig. (), the patch map must include three radial zones because there
are two separatrices there. There are four poloidal zones to represent
four divertor legs there, and together with four poloidal zones
covering the core domain there are total eight poloidal zones. All in
all, there are twenty four patches in this case. Similarly, for each
of the snowflake-like configurations there are twenty four patches, as
illustrated in Figs. ().
