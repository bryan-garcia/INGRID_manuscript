\subsection{Construction of patch-maps}
For a given divertor configuration of interest, INGRID creates a Patch map: the collection of Patch objects that form a partition of the magnetic-topology to be modeled. Construction of a Patch map occurs after topology analysis and, upon completion, enables grid generation. INGRID takes the result of topology analysis to instantiate the appropriate topology class object that contains a collection of configuration-specific line tracing method calls for the construction of the Patch map. The newly constructed object  utilizes user provided psi-boundary values, refined topological reference points, N-S-E-W directions, and target-plate/limiter geometry to construct the final Patch map. Construction of the Patch map relies on the LineTracing class to generate the Line objects that are then trimmed, split, re-ordered, and joined together to define the appropriate Patch objects. Although line tracing routines are magnetic topology specific, there is a general approach taken to construction of Patch map.\\
First, tracing of the primary-separatrix is completed by using N-S-E-W directions obtained from x-point analysis. This is done by line tracing from xpt\_1 NW, NE, SW, and SE. Line tracing from the NW and NE directions form the ``loop" around the core-region, and line tracing originating from the SW and SE directions form the ``legs" of the seperatrix and terminate upon intersection with the appropriate user provided strike point geometry (target-plates or limiter). \\
After tracing of the primary-separatrix, line tracing continues from xpt\_1 with starting points in the N, S, E, W directions and termination occurring upon intersection with the magnetic-topology dependent psi-boundary level provided by the user. The Patch map figures illustrate this configuration dependence. Tracing in the N, S, E, or W direction to a new psi-level enables poloidal line tracing to continue on a new psi-level. This stage in the line tracing is repeated until all Patch boundaries are formed. These boundaries can be seen in the Patch map diagrams provided. Generating a Patch map in the case of two x-points follows the same general framework with the addition of secondary-separatrix line tracing.