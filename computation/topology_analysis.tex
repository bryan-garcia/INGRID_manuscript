\subsection{Topology analysis}
INGRID can perform an automated topology analysis of loaded EFIT data in order to identify the magnetic-topology in question. This topology analysis algorithm is a key feature of INGRID that simplifies user operation of the code and automates much of the tedious grid generation work. In order to achieve a robust topology analysis algorithm, INGRID adopts localized ``compass-directions" at x-points. That is, each x-point has a notion of ``N-S-E-W" in order to aid in the line tracing that is utilized in the topology analysis algorithm. The topology analysis algorithm operates in two main phases: (i) disambiguation of N-S-E-W directions at x-point location(s), (ii) classification of magnetic-topology.\\ \indent
Disambiguation of N-S-E-W ``compass" directions at x-point locations refers to the process of computing starting coordinates for line tracing the INGRID defined N, S, E, and W directions. X-points correspond to saddle-points of the psi-function. When considering a $2\times2$ Hessian matrix evaluated at a saddle-point, two eigenvalues of differing signs and their associated eigenvectors are obtained that indicate directions of increasing psi and decreasing psi. Increasing/decreasing psi also occurs in the direction opposite of the obtained eigenvectors due to being a saddle-surface. Taking an $\epsilon$ step in the direction of all directional-vectors provides two points in the direction of increasing psi, and two points in the direction of decreasing psi. All together, these four points in the general direction of increasing/decreasing psi are ``candidate" N-S-E-W directions for subsequent line tracing. The algorithm for disambiguation of candidate N-S-E-W directions is dependent on the general class of magnetic-topology in question: single-null (SNL), snowflake-plus (SF+), or snowflake-minus (SF-).\\ \indent
Obtaining true N-S-E-W directions in single x-point cases is simple. Using candidate N and S coordinates corresponding to directions of increasing psi, an optimizer is utilized in order to compare the resultant optimal point coordinates to the magnetic-axis coordinates. Since candidate N and S are in basins of a concave/convex regions of the psi function, application of an optimizer to both N and S candidate directions results in one candidate direction leading to the coordinates corresponding to the magnetic-axis; a local maxima of the psi function. Comparing optimizer results obtained from N and S directions to the magnetic-axis coordinates identifies the true N and S directions at the primary x-point. From here, E and W candidate directions are disambiguated by checking (signed) relative angles between true N and S directions. With N-S-E-W directions in hand, classification in the SNL configuration case is complete since INGRID does not treat the special cases of LSN and USN any different than a general SNL.\\ \indent
Cases with two x-points build off the single x-point case by including a preliminary classification scheme to first identify whether the configuration is SF+ or SF-. INGRID first uses the primary separatrix to partition the domain into three distinct regions: core, private-flux, and exterior. SF+ and SF- divertor configurations can be differentiated by observing which of the listed regions the secondary x-point resides in. Should the secondary x-point reside in the private-flux region, the magnetic-topology of interest is SF+. Should the secondary x-point resides in the exterior region, the magnetic-topology is SF-. Once SF+/SF- classification is complete, INGRID continues with analysis of the secondary x-point in order to determine true N-S-E-W directions. The details of analysis, however, are dependent upon whether the configuration is SF+ or SF-.\\ \indent
First, consider the divertor configurations within the SF- category: unbalanced double-null (UDN), SF15, SF45, SF135, and SF165. INGRID first begins by performing an eigenanalysis at the secondary x-point identical to that performed on the primary x-point in the SNL case. After obtaining candidate N-S-E-W coordinates for the secondary x-point, INGRID utilizes a trust-region optimizer on both the candidate N and S directions in order to identify true N and S. A trust-region optimizer constructs a quadratic model to approximate some objective function within a small trust-region and steps according to the model in said region. By fixing the trust-region size to that of the line tracing step size, the trust-region optimizer can act as a controlled and computationally inexpensive method of traveling in a gradient direction. This trick is used by starting at the candidate N and S directions and traveling down the psi gradient until one of the two starting points lands in either the core partition or private-flux partition. True N will land one of the two regions. Should the trust-region search terminate in the core, the EFIT data represents either a UDN, SF15, or SF165 divertor configuration. Should the trust-region search terminates in the private-flux region, the EFIT data represents either an SF45 or SF135 divertor configurations. To finalize magnetic-topology identification, INGRID performs line tracing from true N until intersection with the primary separatrix occurs. The intersection point on the primary separatrix (e.g. inner leg, inner lower core, outer leg, etc)  determines which magnetic-topology is being dealt with.\\ \indent
Next, consider the divertor configurations withing the SF+ category: SF75 and SF105. The SF+ topology does not allow for a trust-region optimization scheme to be relied entirely upon for acceleration of identification. First, candidate N-S-E-W directions for the secondary x-point are obtained by p erforming an eigenanalysis of the secondary x-point and naively assigning N and S to the directions of increasing psi, and E and W to the directions of decreasing psi. It should be noted that this is opposite to that of the SF- and SNL cases. The swap is due to the secondary x-point being located in the private-flux, and ensures consistency of line tracing from true N will intersect the magnetic-axis. From here, line tracing from candidate N and S is executed until primary separatrix intersection from either candidate N and S occurs. Similar to the SF- case, the point of intersection with the primary separatrix determines which magnetic-topology is in question.