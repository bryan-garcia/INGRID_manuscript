\subsection{Topology analysis}

Based on the user input, INGRID seeks a configuration with either one
or two X-points in the domain. If one X-point is required, this is a
single-null (SN) configuration. Although in the tokamak edge plasma
community it is common to distinguish the ``upper single null'' and
the ``lower single null configurations'', for INGRID there is no
distinction between those. Instead of using ``lower'' and ``upper''
for the divertor, and ``inner'' and ``outer'' for target plates,
INGRID uses the notion of the ``compass'' directions
North-South-East-West associated with the primary X-point, and the
North direction is defined to point into the core plasma along the
$\grad \Psi$ direction. Then for a single-null configuration, one of
the plates is in the South-West direction (denoted as a ``West''
plate, and the other one is in the South-East direction, and is
denoted as the ``East'' plate, as illustrated in Fig. ().

If two X-points are required to be in the domain, INGRID performs
further analysis to see in which category the given magnetic
configuration falls. First, it determines whether the secondary
X-point is in the private-flux region or in the common-flux region
(SOL) with respect to the primary separatrix. Next, an orthogonal
projection is constructed from the secondary X-point to the primary
separatrix. If the secondary X-point is in the private-flux region
with respect to the primary separatrix, this can be either SF105 or
SF75 configuration.  Then an orthogonal projection is constructed from
the secondary X-point to the primary separatrix, and based on the
location of this projection the specific configuration is identified.

In case the secondary X-point is in the common-flux region with
respect to the primary separatrix, and the secondary X-point and the
primary X-point are on opposite sides of the midplane (the horizontal
plane through the magnetic axis) this is the unbalanced double null
(UDN) configuration, see Fig. ().

In case the secondary X-point is in the common-flux region with
respect to the primary separatrix, and the secondary X-point and the
primary X-point are on the same side of the midplane this can be on of
four possible cases: SF15, SF45, SF135, or SF165 configurations, see
Figs (). Then an orthogonal projection is constructed from the
secondary X-point to the primary separatrix, and based on the location
of this projection the specific configuration is identified.
