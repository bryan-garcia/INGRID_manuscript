\subsection{Topology analysis}
Here we detail how we detail INGRID's disambiguation of N-S-E-W directions and INGRID's classification algorithm.\\ \indent
When locally inspecting the psi function around an x-point, we see we are dealing with a saddle surface, with the x-point being the saddle-point. Since this holds for any x-point of interest, candidate N-S-E-W directions are obtained by eigen-analysis of the $2\times2$ Hessian matrix evaluated at the saddle-point. This eigen-analysis yields two eigenvalues and their associated eigenvectors. Due to the saddle-point, these two eigenvalues are of differing signs with the eigenvector associated with the positive eigenvalue indicating a direction of increasing psi. On the other hand, the eigenvector associated with the negative eigenvalue indicates one direction of decreasing psi. We note that INGRID normalizes the psi-function to the magnetic-axis and primary x-point so that $\psi = 0$ and $\psi=1$ at each $(r,z)$ coordinate, respectively. Due to being a saddle-surface, we must take into account the increasing/decreasing psi that occurs in the direction opposite of the obtained eigenvectors. Taking an $\epsilon$ step in the direction of all four vectors provides us with two points in the direction of increasing psi, and two points in the direction of decreasing psi. These are our ``candidate" N-S-E-W directions.\\ \indent
Disambiguation of these directions in the SNL case is quick. With candidate N and S directions, we apply a constrained optimizer to each Point and compare the resultant optimal point coordinates to the magnetic-axis coordinates. Since candidate N and S are in basins of a convex region of the psi function, the optimizer will indeed produce a result corresponding to either the magnetic-axis sink or private-flux sink. This step establishes true N and S directions, and allows us to further disambiguate the E and W candidate directions by checking the signs of the relative angles between the two sets of vectors.\\ \indent
As for classification of SNL configurations, we find that this is a trivial case. Since INGRID does not treat the LSN and USN differently internally, the user's indication within the parameter file that the number of x-points is only one is sufficient for INGRID identification of SNL configurations.\\ \indent
Cases with two x-points require more work from INGRID for both disambiguation and magnetic topology classification. INGRID accomplishes this task by first identifying whether the configuration is SF+ or SF-. To do so, INGRID utilizes the primary separatrix to partition the domain into three distinct regions: the core, the private-flux, and the exterior (complement of the union of the core and private-flux). SF+ and SF- divertor configurations can be differentiated by noting which region the secondary x-point resides in. Should the secondary x-point reside in the private-flux region obtained above, the divertor configuration of interest is SF+. On the other hand, should the secondary x-point reside in the exterior region defined above, the divertor configuration is SF-. From here INGRID executes a further analysis consisting of N-S-E-W and LineTracing methods to determine the specific instance of SF+ or SF- the EFIT data represents. In general, this final step is to determine where line tracing from the true N direction of the secondary x-point intersects the primary separatrix.\\ \indent
First let us consider the divertor configurations within the SF- collection. These are the unbalanced double-null (UDN), SF15, SF45, SF135, and SF165 divertor configurations. INGRID first begins by performing an eigen-analysis of the secondary x-point. Similar to the SNL case, candidate N and S directions are in directions of increasing psi, and candidate E and W directions are in directions of decreasing psi. To determine true N and S, INGRID this time utilizes a trust-region optimization scheme on both the candidate N and S directions until one of the two starting points lands in either the core partition or private-flux partition. This termination criteria is to both determine the true N direction of the secondary x-point, and simplify our classification algorithm. Should the trust-region search terminate in the core, we are dealing with either UDN, SF15, or SF165 divertor configurations. When the trust-region search terminates in the private-flux region, we are dealing with either SF45 or SF 135 divertor configurations. At this point, INGRID can perform LineTracing from true N to determine the point of intersection with the primary separatrix.\\ \indent
When dealing with SF+ configurations, we do not employ a trust-region search to aid in our classification algorithm. With only two cases to consider (SF75 and SF105), we can rely entirely on the LineTracing class. As in all other cases, we first obtain candidate N-S-E-W directions for the secondary x-point. For SF+ cases, we assign candidate N and S to the directions of increasing psi, and E and W to the directions of decreasing psi. This swap is due to the secondary x-point being located in the private-flux, and ensures line tracing from true N will intersect the magnetic-axis. Performing an eigen-analysis of the secondary x-point, we obtain the candidate N-S-E-W directions and simple line trace from candidate N and S while searching for intersection with the primary separatrix. Like in the SF- case, we determine SF75 or SF105 depending on the intersection point on the primary separatrix.\\ \indent
The above divertor configuration identification algorithm is handled internally and does not require the user to intervene at any point. This encapsulation simplifies the grid generation process by only requiring the user to indicate the number of x-points to consider when generating a grid.
