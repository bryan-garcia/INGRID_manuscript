\subsection{\label{sec:level2}Bicubic interpolation and line tracing}
INGRID utilizes bicubic interpolation for obtaining psi values between the provided data points of a provided magnetic equilibrium eqdsk input file (EFIT file). INGRID calculations involve solving differential equations that require a continuous $\psi, \psi'_{z}, \psi'_{r}, \psi''_{rz}$. Bicubic interpolation is robust, preserves continuity and smoothness at the edges of cells, and provides the accuracy required for modeling divertor configurations with small poloidal magnetic field values and low neqdsk resolution. For our use of bicubic interpolation, derivatives $\frac{d \psi}{dx}$ and $\frac{d\psi}{dz}$ on EFIT grid by 2D finite-difference. After that, $\psi$ is reconstructed inside the cell defined by four EFIT grid points. Two useful properties of bicubic interpolation are: (i) values of the function and the specified derivatives are reproduced exactly on the grid points, and (ii) values of the function and the specified derivatives change continuously as the interpolating point crosses from one grid square to another \cite{Press_1992}. INGRID utilizes bicubic interpolation for computational tasks such as line tracing. Line Tracing using the LSODA integrator in SciPy used to generate poloidal (eq. 1) and radial (eq. 2) surfaces.

\beq
%
\dot{R} = -\frac{1}{R}\psi_{z} \quad \dot{Z} = \frac{1}{R}\psi_{r}
%
\eeq

\beq
%
\dot{R} = \frac{1}{R}\psi_{z} \quad \dot{Z} = \frac{1}{R}\psi_{z}
%
\eeq

\noindent
Line tracing capabilities also allow for computation of lines with constant arbitrary slope. These line tracing capabilities allow for the development of Patch map generation scripts. A variety of termination criteria for line tracing are available, and include intersection criteria such as $(r,z)$ intersection, $\psi$ value intersection, and intersection with arbitrary INGRID ``curves" (piecewise linear geometries).