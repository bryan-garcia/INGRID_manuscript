\subsection{\label{sec:level2}Magnetic field interpolation}

Input data for grid generation come in the form of the poloidal
magnetic flux $\Psi$ sampled on rectilinear grid in $R,Z$, from an MHD
reconstruction code such as EFIT \cite{Lao_1985} or TEQ \cite{}. For
obtaining the values of the poloidal magnetic flux $\Psi$ and its
derivatives between data points of a provided magnetic equilibrium,
INGRID utilizes bicubic interpolation \cite{Press_1992}. Bicubic
interpolation needs function values and derivatives to be provided on
the original grid, and for our use of bicubic interpolation, the
derivatives $\partial_Z \Psi, \partial_R \Psi, \partial^2_{R,Z} \Psi$
are first evaluated at the original grid points by finite difference.


Flux surfaces are reconstructed in INGRID by integrating the ODEs,

\beqar
%
\dot{R} = -\frac{1}{R}\psi_{z} \\
\dot{Z} = \frac{1}{R}\psi_{r}
%
\eeqar

The bicubic interpolation guarantees smoothness of $\Psi$ at the edges
of cells of the original $R,Z$ grid, so the resulting flux surfaces
are smooth.



%using the LSODA integrator in SciPy.

%The properties of bicubic interpolation are: (i) values of the
%function and the specified derivatives are reproduced exactly on the
%grid points, and (ii) values of the function and the specified
%derivatives change continuously as the interpolating point crosses
%from one grid square to another \cite{Press_1992}. Therefore flux
%surfaces, based on the bicubic interpolation, are smooth.

%Since bicubic interpolation guarantees continuity of $\Psi'_{z},
%\Psi'_{r}$ as the interpolating point crosses from one grid square to
%another \cite{Press_1992}, flux surfaces based on the bicubic
%interpolation are smooth.

%\beq
%%
%\dot{R} = \frac{1}{R}\psi_{z} \quad \dot{Z} = \frac{1}{R}\psi_{z}
%%
%\eeq

%\noindent
%Line tracing capabilities also allow for computation of lines with
%constant arbitrary slope. These line tracing capabilities allow for
%the development of Patch map generation scripts. A variety of
%termination criteria for line tracing are available, and include
%intersection criteria such as $(r,z)$ intersection, $\psi$ value
%intersection, and intersection with arbitrary INGRID ``curves"
%(piecewise linear geometries).
