\subsection{\label{sec:level2}Bicubic interpolation}
INGRID utilizes bicubic interpolation for obtaining psi values between the provided data points of a provided magnetic equilibrium eqdsk input file (EFIT file). This choice of interpolation is robust and provides the accuracy required for modeling divertor configurations with small poloidal magnetic field values and low neqdsk resolution. The Bicubic class is composed with the EfitData class: a class responsible for the management of EFIT data provided to INGRID. The EfitData class is utilized in conjunction with the LineTracing class to compute radial lines and poloidal lines. 

\subsection{\label{sec:level2}Line tracing}
The creation of the Line objects from interpolated EFIT data that are utilized for grid generation is handled by INGRID's LineTracing class. All line tracing that occurs during operation is handled by an instantiation of the LineTracing class that is managed by the global Ingrid object users interface with either via GUI or scripts. The LineTracing class can line trace parallel and perpendicular to psi-surfaces, as well as line trace with a constant arbitrary slope. These line tracing options have proved sufficient for current grid generation needs, but additional capabilities can easily be added. The LineTracing class has robust line tracing termination criteria that simplifies development of Patch map generation scripts. Termination criteria is specified as a method argument and supports a variety of of intersection criteria such as Point intersection, psi-level intersection, and Line object(s) intersection(s). All ODE numerical integration for line tracing purposes currently utilize the scipy.integrate.LSODA integrator. Users of INGRID can configure integrator settings within the parameter file.
% The LineTracing class can also be configured in order to provide a ``live" plot of line tracing for developers and debugging purposes. 