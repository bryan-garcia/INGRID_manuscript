\subsection{\label{sec:level2} ``Divide and conquer'' strategy}

The basic idea of the INGRID workflow is: (i) first construct a
``skeleton'' grid (also called a ``patch-map'') which corresponds to
the geometry of the magnetic field in hand and consists of a small
size of quadrilateral patches, and (ii) put a subgrid on each
patch. This is illustrated in Fig. (\ref{fig:patchmap_subgrid}) where
a patch-map is shown with one of the patches covered with a subgrid.

As explained further, the basic steps of the calculation are: (i)
import given magnetic field data defined on (r,z) grid, and construct
an interpolation fit for it; (ii) find the reference points in the
data (X-points, magnetic axis); (iii) analyze the topology of the
magnetic field; (iv) construct a ``skeleton'' grid; and (v) put
subgrids in each ``patch'' to produce the final grid.
