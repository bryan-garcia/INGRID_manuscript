\subsection{\label{sec:level2} ``Divide and conquer'' strategy}

The INGRID workflow includes two main steps: (i) constructing a
``skeleton grid'' (also called further a ``patch-map'') which
corresponds to the geometry of the magnetic field in hand and consists
of a small number of quadrilateral patches; and (ii) putting a subgrid
on each of the patches. This is illustrated in
Fig. (\ref{fig:patchmap_subgrid}) where a patch-map is shown with one
of the patches covered with a subgrid.

The skeleton grid is constructed as a smallest possible grid to be
aligned with the given magnetic field and respecting the magnetic
field topology. The geometry of magnetic flux surfaces, in particular
the X-points and the magnetic axis, and the geometry of plasma facing
material surfaces all together define a patch map. For calculating a
subgrid, the code divides each quarilateral patch into a number of
radial and poloidal zones according to user-provided input. Finally,
all subgrids are joined together to produce the global grid.
